\documentclass[12pt]{article}
\usepackage{graphics,graphicx,array,color}

\headheight=0in
\headsep=0.2in
\topskip=0in
\setlength{\oddsidemargin}{-0.5in}
\setlength{\evensidemargin}{0cm}
\setlength{\topmargin}{-0.5in}
\setlength{\textwidth}{7.0in}
\setlength{\textheight}{10.0in}

\begin{document}
\hfill \includegraphics[width=0.35\textwidth]{siena_phys_astro_print_crop.jpg}

\vspace{0.1cm}
\hspace{0.1in}{\Large \bf Physics BS$^{1}$ (Applied Physics) 4-Year Schedule}
\vspace{0.1cm}

    \small
\vspace*{-5mm}
\begin{table}[h!]
\begin{center}
{\renewcommand{\arraystretch}{1.2}
\begin{tabular*}{0.9\textwidth}{@{\extracolsep{\fill}}lclc}
%\hline
 & \\
{\Large \textbf{Fall Year 1}} & & {\Large \textbf{Spring Year 1}} & \\
\hline
{\em PHYS~130: General Physics I}\,$^{2}$ (CDN)& 4 & {\em PHYS~140: General Physics II}     & 4 \\
PHYS~132: General Physics Review           & 0 & PHYS~142: General Physics Review & 0\\
{\em MATH~110: Calculus I}\,$^{3}$ (CDQ)   & 4 & {\em MATH~120: Calculus II}            & 4 \\
FYSN~100: First-Year Seminar               & 3 & FYSN~101: First-Year Seminar     & 3\\
SCDV~020: Intro to Engineering             & 1 & Creative Arts Core (CDE)         & 3 \\
CSIS~200: Software Tools for Physicists    & 3 &  \\

 & \\
{\Large \textbf{Fall Year 2}} & & {\Large \textbf{Spring Year 2}} & \\
\hline
{\em PHYS~220: Modern Physics}\,$^{4}$ (CFN)& 4 &\textcolor{red}{\em SCDV~???: Applied Computing} & 3  \\ 
{\em SCDV 230: Electronic Instrumentation} & 4 &MATH~325: Differential Equations & 3 \\ 
{\em \textcolor{red}{ENGR~1100: IEA}}& 4 & PHYS~260: Thermal Physics        & 3 \\
Social Science Core (CDS)            & 3 & {\bf and} {\em PHYS~???: Statics and Fluids} & 1 \\
                                          &   & {\bf or} \textcolor{red}{\em ENGR2250: Thermal and Fluids Engineering} & 4 \\
                                          &   &   Social Justice Franciscan Core (CFJ) & 3 \\

 & \\
{\Large \textbf{Fall Year 3}} & & {\Large \textbf{Spring Year 3}} & \\
\hline
{\em PHYS~310: Mechanics I}         & 4 & {\em PHYS~410: Electromagnetic Theory}  & 4 \\
{\em CHEM~110: General Chemistry I} & 4 & MATH~230: Linear Algebra    & 3 \\
MATH~371: Probability for Statistics   & 3 & Philosophy Core (CDP)             & 3 \\
History Core (CDH)                  & 3 & {\em PHYS~370: Experimental Techniques}\,$^{5}$ & 2 \\
 & \\
{\Large \textbf{Fall Year 4}} & & {\Large \textbf{Spring Year 4}} & \\
\hline
{\em PHYS~470: Advanced Lab I} & 1 & {\em PHYS~472: Advanced Lab II}  & 1 \\
{\em \textcolor{red}{???: Upper-level applied course}} & 3 & {\em \textcolor{red}{???: Upper-level applied course}} & 3 \\
English Core (CDA)                         & 3 & Religion Core (CDR)               & 3 \\
Heritage Franciscan Core (CFH)             & 3 &  Franciscan Diversity Core (CFD)  & 3 \\
\end{tabular*}
}
\end{center}
\end{table}

\vspace*{-4mm}
\hspace{0.1in}$^{1}$A minimum of 120 credit-hours is required to
graduate (average 15 credit-hours per semester).  \hspace*{0.42in}Courses in italics have a lab
component (generally indicating a larger time commitment). 

\hspace{0.1in}$^{2}$General Physics satisfies the Natural Science Core (CDN) requirement.

\hspace{0.1in}$^{3}$Calculus satisfies the Quantitative Core (CDQ) requirement.

\hspace{0.1in}$^{4}$Modern Physics satisfies the Natural World Franciscan Core
(CFN) requirement.

\hspace{0.1in}$^{5}$This requirement can be satisfied by taking {\em ASTR~380:
  Observational Astronomy} (a 3-credit \hspace*{0.42in}course offered in the
fall), or by completing the Astrophysics Minor.

\end{document}
