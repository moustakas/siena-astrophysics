\documentclass[12pt]{article}
\usepackage{graphics,graphicx,array}

\headheight=0in
\headsep=0.2in
\topskip=0in
\setlength{\oddsidemargin}{-0.5in}
\setlength{\evensidemargin}{0cm}
\setlength{\topmargin}{-0.5in}
\setlength{\textwidth}{7.0in}
\setlength{\textheight}{10.0in}

\begin{document}
\hfill \includegraphics[width=0.40\textwidth]{siena_phys_astro_print_crop.jpg}

\vspace{0.7cm}
\hspace{0.1in}{\Large \bf Possible Physics BS$^{1}$ 4-Year Course Schedule}
\vspace{0.3cm}

\vspace*{-5mm}
\begin{table}[h!]
\begin{center}
{\renewcommand{\arraystretch}{1.2}
\begin{tabular*}{0.9\textwidth}{@{\extracolsep{\fill}}lclc}
%\hline
 & \\
{\Large \textbf{Fall Year 1}} & & {\Large \textbf{Spring Year 1}} & \\
\hline
PHYS~130: General Physics I$^{2}$  & 4 & PHYS~140: General Physics II     & 4 \\
PHYS~132: General Physics Review  & 0 & PHYS~142: General Physics Review & 0\\
MATH~110: Calculus I$^{3}$        & 4 & MATH~120: Calculus II            & 4 \\
FYSN~100: First-Year Seminar      & 3 & FYSN~101: First-Year Seminar     & 3\\
CSIS~200: Software Tools for Physicists & 3                              & \\ 

 & \\
{\Large \textbf{Fall Year 2}} & & {\Large \textbf{Spring Year 2}} & \\
\hline
PHYS~220: Modern Physics$^{4}$      & 4 & PHYS~260: Thermal Physics        & 3 \\
SCDV 230: Electronic Instrumentation & 4 & PHYS~250: Computational Physics  & 3  \\
MATH~210: Calculus III               & 4 & MATH~325: Differential Equations & 3 \\
Arts Core (CDA)                      & 3 & English Core (CDE)               & 3 \\
                                     &   & Franciscan Diversity Core (CFD)  & 3 \\

 & \\
{\Large \textbf{Fall Year 3}} & & {\Large \textbf{Spring Year 3}} & \\
\hline
PHYS~310: Mechanics I             & 4 & PHYS~410: Electromagnetic Theory  & 4 \\
MATH~330: Intro to Applied Math I & 3 & MATH~230: Linear Algebra$^{5}$    & 3 \\
History Core (CDH)                & 3 & Philosophy Core (CDP)             & 3 \\
CHEM~110: General Chemistry I     & 4 & PHYS~370: Experimental Techniques & 2 \\
                                  &   & \hspace{0.5mm} \textbf{or} PHYS~380: Observational Astronomy & 3 \\
                                  &   & \hspace{0.5mm} \textbf{or} Minor in Astrophysics     &  \\

 & \\
{\Large \textbf{Fall Year 4}} & & {\Large \textbf{Spring Year 4}} & \\
\hline
PHYS~470: Advanced Lab I       & 1 & PHYS~472: Advanced Lab II  & 1 \\
PHYS~440: Quantum Physics      & 3 & Physics Elective           & 3 \\
Religion Core (CDR)            & 3 & Social Science Core (CDS) & 3 \\
Heritage Franciscan Core (CFH) & 3 & Social Justice Franciscan Core (CFJ) & 3 \\
\hline
\end{tabular*}
}
\end{center}
\end{table}

\vspace*{-4mm}
\hspace{0.1in}$^{1}$A minimum of 120 credit-hours is required to
graduate (average 15 credit-hours per semester).

\hspace{0.1in}$^{2}$General Physics satisfies your Natural Science core (CDN).

\hspace{0.1in}$^{4}$Calculus satisfies your Quantitative core (CDQ).

\hspace{0.1in}$^{4}$Modern Physics satisfies your Natural World Franciscan core
(CFN). 

\hspace{0.1in}$^{5}$This sixth math class gives you a Mathematics Minor (which
must be declared).

\end{document}
