\documentclass[12pt]{article}
\usepackage{graphics,graphicx,array}

\headheight=0in
\headsep=0.2in
\topskip=0in
\setlength{\oddsidemargin}{-0.5in}
\setlength{\evensidemargin}{0cm}
\setlength{\topmargin}{-0.5in}
\setlength{\textwidth}{7.0in}
\setlength{\textheight}{10.0in}

\begin{document}
\hfill \includegraphics[width=0.35\textwidth]{siena_phys_astro_print_crop.jpg}

\vspace{0.1cm}
\hspace{0.1in}{\Large \bf Physics BS$^{1}$ (with Astrophysics Minor) 4-Year Schedule}
\vspace{0.3cm}

\vspace*{-5mm}
\begin{table}[h!]
\begin{center}
{\renewcommand{\arraystretch}{1.2}
\begin{tabular*}{0.9\textwidth}{@{\extracolsep{\fill}}lclc}
%\hline
 & \\
{\Large \textbf{Fall Year 1}} & & {\Large \textbf{Spring Year 1}} & \\
\hline
{\em PHYS~130: General Physics I}\,$^{2}$  & 4 & {\em PHYS~140: General Physics II}     & 4 \\
PHYS~132: General Physics Review           & 0 & PHYS~142: General Physics Review & 0\\
{\em MATH~110: Calculus I}\,$^{3}$         & 4 & {\em MATH~120: Calculus II}            & 4 \\
FYSN~100: First-Year Seminar               & 3 & FYSN~101: First-Year Seminar     & 3\\
CSIS~200: Software Tools for Physicists    & 3 & ASTR~101: Introductory Astronomy for Scientists & 3 \\

 & \\
{\Large \textbf{Fall Year 2}} & & {\Large \textbf{Spring Year 2}} & \\
\hline
{\em PHYS~220: Modern Physics}\,$^{4}$     & 4 & PHYS~260: Thermal Physics        & 3 \\
{\em SCDV 230: Electronic Instrumentation} & 4 & {\em PHYS~250: Computational Physics} & 3  \\
{\em MATH~210: Calculus III}               & 4 & MATH~325: Differential Equations & 3 \\
ASTR~390: Principles of Astrophysics~I     & 3 & ASTR~392: Principles of Astrophysics~II  & 3 \\
                                           &   & Franciscan Diversity Core (CFD)  & 3 \\

 & \\
{\Large \textbf{Fall Year 3}} & & {\Large \textbf{Spring Year 3}} & \\
\hline
{\em PHYS~310: Mechanics I}         & 4 & {\em PHYS~410: Electromagnetic Theory}  & 4 \\
{\em CHEM~110: General Chemistry I} & 4 & MATH~230: Linear Algebra$^{5}$    & 3 \\
MATH~330: Intro to Applied Math I   & 3 & ASTR~332: Astrophysics Seminar~II$^{6}$     & 2 \\
{\em ASTR~380: Observational Astronomy} & 3 & Philosophy Core (CDP)             & 3 \\
                                    &   & Social Justice Franciscan Core (CFJ) & 3 \\

& \\
{\Large \textbf{Fall Year 4}} & & {\Large \textbf{Spring Year 4}} & \\
\hline
{\em PHYS~470: Advanced Lab I} & 1 & {\em PHYS~472: Advanced Lab II}  & 1 \\
PHYS~440: Quantum Physics      & 3 & Physics Elective           & 3 \\
Heritage Franciscan Core (CFH) & 3 & Social Science Core (CDS)  & 3 \\
Creative Arts Core (CDA)       & 3 & Religion Core (CDR)        & 3 \\
English Core (CDE)             & 3 & History Core (CDH)         & 3 \\
\hline
\end{tabular*}
}
\end{center}
\end{table}

\vspace*{-4mm}
\hspace{0.1in}$^{1}$A minimum of 120 credit-hours is required to
graduate (average 15 credit-hours per semester).  \hspace*{0.42in}Courses in italics have a lab
component (generally indicating a larger time commitment). 

\hspace{0.1in}$^{2}$General Physics satisfies the Natural Science Core (CDN) requirement.

\hspace{0.1in}$^{3}$Calculus satisfies the Quantitative Core (CDQ) requirement.

\hspace{0.1in}$^{4}$Modern Physics satisfies the Natural World Franciscan Core
(CFN) requirement.

\hspace{0.1in}$^{5}$This sixth math class gives you a Mathematics Minor (which
must be declared).

\hspace{0.1in}$^{6}$Alternatively, ASTR~330: Astrophysics Seminar~I may be taken
in the fall (if offered).

\end{document}
