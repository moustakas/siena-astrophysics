\documentclass[12pt]{article}
\usepackage{graphics,graphicx,array,url}

\headheight=0in
\headsep=0.2in
\topskip=0in
\setlength{\oddsidemargin}{-0.5in}
\setlength{\evensidemargin}{0cm}
\setlength{\topmargin}{-0.5in}
\setlength{\textwidth}{7.0in}
\setlength{\textheight}{9in}

\begin{document}
\hfill \includegraphics[width=0.35\textwidth]{siena_phys_astro_print_crop.jpg}

\vspace{0.2cm}
\begin{center}
{\LARGE {\bf Frequently Asked Questions}} \\
\medskip
{\Large {\bf 4/1 Program with Clarkson University}} 
\end{center}
\vspace{0.3cm}

%\noindent This is good.

\begin{itemize}
\item{{\bf {\em What is this document? Where can I get for more information?}} 

This document is meant to be an unofficial guide for students at Siena College
\emph{considering} the 4/1 Program; it is not a replacement for speaking at
length with your academic advisor and with the Dean of the School of Engineering
and Computer Science at Clarkson University, Robert Kozik (bkozik@clarkson.edu).
At Siena you may also reach out to Prof. John Moustakas (jmoustakas@siena.edu)
in the Physics Department if you have specific questions not answered below.}

\item{{\bf {\em What is the 4/1 program?}}

The 4/1 Program is an articulation between Siena College and Clarkson University
which allows undergraduate students pursuing their BS in Physics, Environmental
Science, Computer Science, or Mathematics to take graduate-level courses at the
Clarkson University Capital Region Campus in Schenectady, NY, giving them a jump
on obtaining their engineering masters degree in just one additional year
(hence, the ``4/1 Program'').  You may take up to \underline{three} courses at
Clarkson as part of your Siena tuition (i.e., without paying any additional
fees).  In addition, the courses you take at Clarkson may also satisfy your
upper-level major course requirements, although that depends on your
department---see your advisor for details.}

%In the physics department these courses also count toward your
%required upper-level physics courses (as PHYS400).  See your advisor for
%details.} 

\item{{\bf {\em What engineering degrees are available?}}

  The three Master of Science (M.S.) programs currently available to students
  are {\em Electrical Engineering} (MSEE), {\em Engineering and Management
    Systems} (MSEM), and {\em Energy Systems} (MSE).  Quoting from the program
  catalog, the MSEE program ``explores technologies and industry opportunities
  in modern electric machinery, modeling and control of power electronics.''
  The MSEM program integrates ``engineering and information systems technologies
  with the core components of an MBA.''  And the MSE degree enables ``students
  to integrate (1) mechanical/electrical energy related courses, (2) mechanical
  and electrical fundamental discipline courses, and (3) non-technical courses
  regarding the impact of environmental, economic, and regulatory issues on
  energy.''}

\item{{\bf {\em When can I start taking classes at Clarkson?}}

  Students can start taking classes as soon as they have sufficient mathematics
  preparation, typically in their junior or senior year.  Although these are not
  formal prerequisites, it is strongly recommended that students have completed
  all three semesters of calculus, differential equations, and either linear
  algebra or applied mathematics (ideally, both) before tackling these
  graduate-level courses.  You should also have some experience with {\tt
    Matlab}/{\tt Simulink}.}

\item{{\bf {\em When are the academic terms at Clarkson?}}

  Unlike Siena, Clarkson is on a 10-week trimester system.  The fall, winter,
  and spring trimesters are roughly from early September to late November, early
  January to mid-March, and early April to early June, respectively.  (Please
  note that only the fall term is aligned with the housing term at Siena;
  therefore, if you do not live locally and need to take winter and/or spring
  courses you need to consider where you can live while taking courses at
  Clarkson.)  In addition there is typically a summer session of selective
  classes from mid-June to the end of August.  The classes typically take place
  during the week and in the evening.  The average class size is around 10
  students.}

\item{{\bf {\em What classes are offered?  What are the prerequisites?}}

  The courses available varies somewhat from year-to-year, so you should speak
  with your advisor about which courses are available and whether you are ready
  to start taking courses at Clarkson.  However, the \emph{first} course Siena
  students typically take regardless of which program they are pursuing is {\em
    EER 522: Linear Control Systems}.}

%\item{{\bf {\em How do I apply to the program?  What are the minimum
%      requirements?}} 
%
%To apply to the graduate program at Clarkson you must have a minimum 3.0 GPA and
%you must submit an official \underline{Application
%  Form}\footnote{http://www.clarkson.edu/artsandsci/documents/graddocuments/CompleteApplicationPacket.pdf}
%but with some important differences specifically for Siena students.  In
%addition to the application form, you will also need to submit an official
%transcript, a personal statement, and three letters of recommendation; the
%application fee is waived for Siena students.  Note that it is possible---and
%indeed recommended---that you take at least one class \emph{before} officially
%applying to Clarkson, so you can get a better idea of the courses and the
%program itself.  In that case you should get in contact with Dean Kozik about
%what documentation he needs to receive (typically an unofficial transcript and
%an application form).}
  
\item{{\bf {\em How do I apply to the program?  How do I register for classes?
      What are the minimum requirements?}}

In order to apply to the program you must submit an \underline{Application
  Form}\footnote{http://www.clarkson.edu/artsandsci/documents/graddocuments/CompleteApplicationPacket.pdf},
an official transcript, a personal statement, proof of immunizations, and two
(not three) letters of recommendation; the application fee is waived for Siena
students.  Note that a minimum 3.0 GPA is also required.  It is possible---and
indeed recommended---that you take at least one class \emph{before} officially
applying to Clarkson, so you can get a better idea of the courses and the
program itself.  In that case you should get in contact with Dean Kozik about
what documentation must be received in order to take a trial class.  

In order to register for classes you need to fill out the
\underline{Registration
  Form}\footnote{\url{http://clarkson.edu/sas/forms/crc_registration_siena.pdf}}
specifically designed for Siena students.  The second page of this registration
form has detailed instructions that you should follow carefully, but basically
the form will need to be signed by your acadmemic advisor, the Dean of the
School of Science, and the Siena College registrar (Jim Serbalik, Siena Hall
102) before it is sent to Clarkson.  Be sure to get these signatures well in
advance of the start of classes!}

\item{{\bf {\em How do I get credit for these classes on my CAPP report?}}

The classes you take at Clarkson as part of the 4/1 Program are treated by the
Siena registrar as if they were courses you took at Siena itself.  Therefore,
once you complete the course at Clarkson and your grade has been submitted, a
transcript from Clarkson will be sent to Siena so the grade and course can
appear on your transcript and CAPP report.  This is supposed to happen
automatically between the two institutions (without a fee!); however, it
behooves you to follow up in person, by calling, or by emailing---and with your
advisor---to make sure all the necessary information has been received and that
your CAPP report looks correct.  For physics students, courses taken at Clarkson
appear on their transcript and CAPP report as PHYS400 ({\em Special Topics in
  Physics}).}

\item{{\bf {\em Are there research opportunities?}}

Unfortunately there are no research opportunities, although you should speak
with Dean Kozik early and often about possible paid and unpaid internships.}

%\item{{\bf {\em How do I get to Clarkson and where can I park?}}
%Answer}

%\item{{\bf {\em Where can I read the articulation agreement between Siena and
%      Clarkson?}}
%link}

\item{{\bf {\em What happened to Union Graduate College?}}

  In January 2016 Union Graduate College became the Clarkson University Capital
  Region Campus, a satellite campus of Clarkson University which is located in
  Potsdam, NY.}
\end{itemize}
  
\end{document}
